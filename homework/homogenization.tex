\documentclass{article}
\usepackage[utf8]{inputenc}

\usepackage{amsmath}
\usepackage{braket}
\usepackage{stmaryrd}
\usepackage[margin=1in]{geometry}

\newcommand{\fluctint}[1]{\llbracket #1 \rrbracket}

\begin{document}

1.  Write a spectral solver for the homogenized equations we derived in class:
\begin{align}
    \overline{K^{-1}} \overline{p}_t + \overline{u}_x & = \delta \alpha \hat{c}^2 \overline{u}_{xx} \\
    \overline{\rho} \  \overline{u}_t + \overline{p}_x & = -\delta \alpha \hat{c}^2 \overline{p}_{xx}
\end{align}
where
$$
    \alpha = \overline{K^{-1}\fluctint{\rho}}
$$
with
$$
    \fluctint{f}(y) = \int_0^y\left\{f(\xi)\right\}d\xi  - \int_0^1 \int_0^\tau \left\{f(\xi)\right\}d\xi d\tau.
$$
where $\left\{f(y)\right\} = f(y) - \int_0^1 f$,
and $\delta$ is the period of the coefficient functions $K(x), \rho(x)$.

Since these equations are linear, you don't need the pseudospectral approach and you don't need to
discretize in time.
\vspace{0.3in}

2.  Compare the solution from your spectral solver with the direct solution of the variable-coefficient wave
equation.  For the latter, you may use the code in the notebook presented in class.  How does the agreement between
these two change as you:
\begin{itemize}
    \item Vary the final time?
    \item Vary the length of the spatial period?
    \item Vary the "wavelength" of the initial data?
    \item Vary the amount of variation in the impedance: $Z(x) = \sqrt{K(x) \rho(x)}$?
\end{itemize}

\vspace{0.3in}
\noindent {\bf If you're interested, here are some other questions you could consider:}
\vspace{0.3in}

3.  Work out the next order terms using perturbation theory.  Add these to your spectral solver.  How much does the agreement between
the homogenized system and the variable-coefficient system improve?

Can you conjecture the general form of the corrections that will appear at subsequent orders?
\vspace{0.3in}

4.  In class, we wrote the homogenized system (without exchanging $t$-derivatives for $x$-derivatives) 
as a set of 4 first-order equations (see the notes).  What is the dispersion relation for this system, and how does it differ
from the dispersion relation we derived in class?

Write a solver for this system using either operator splitting or an exponential method, and compare the results
it gives with those obtained in part (2) above.

\end{document}
